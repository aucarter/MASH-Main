\documentclass{article}
\usepackage{mathtools}
\usepackage{csquotes}
\renewcommand{\baselinestretch}{1.2} 
\newcommand{\eg}{{\em e.g., }}
\newcommand{\ie}{{\em i.e., }}
\newcommand{\fp}{$\left\{f\right\}$}
\newcommand{\lp}{$\left\{l\right\}$}
\newcommand{\ssp}{$\left\{s\right\}$}
\newcommand{\mm}{$\left\{m\right\}$}
\newcommand{\ND}{{\bf NEEDS DOCUMENTATION.}}
\newcommand{\Bbout}{{\bf [B]}}
\newcommand{\Fbout}{{\bf [F]}}
\newcommand{\Rbout}{{\bf [R]}}
\newcommand{\Lbout}{{\bf [L]}}
\newcommand{\Obout}{{\bf [O]}}
\newcommand{\Sbout}{{\bf [S]}}
\newcommand{\Mbout}{{\bf [M]}}
\newcommand{\Ebout}{{\bf [E]}}
\newcommand{\SSbout}{{\bf [Ss]}}
\newcommand{\MSbout}{{\bf [Ms]}}
\newcommand{\ESbout}{{\bf [Es]}}
\newcommand{\Dead}{{\bf [D]}}
%\newcommand{\queue}{{\verb1QUEUE1 }}
\newcommand{\demo}{{{\bf \em The Demo}}}
\newcommand{\minmod}{{{\bf \em The Minimal Model}}}
\newcommand{\planned}{{{\bf \em NOTE: This feature has not been fully implemented. It is planned for a future release of MBITES.}}}


\usepackage{Sweave}
\begin{document}

\begin{centering}
\Huge{MBITES}\\
\large{Mosquito Bout-based and Individual-based \\ Transmission Ecology Simulator}\\
\vspace{0.2in}
\huge{A User's Guide} \\
\large{Version 0.1, January 16, 2018} \\
\vspace{0.3in} 
\large{\underline {MBITES Development Team:}

\vspace{0.1in} 

Sean Wu, Hector Sanchez, Qian Zhang, John Henry, Daniel Citron, Amit Verma, Alec Georgoff, John Marshall, Arnaud Le Menach, David L Smith\\}

\vspace{0.15in} 

\large{\underline {Entomological Experts:}

\vspace{0.1in} 

Anne Wilson, Steven W. Lindsay, Thomas W. Scott\\}

\end{centering}

%\newcommand{\mbites}{{\cal MBITES}}

\Sconcordance{concordance:MBITES-UsersGuide.tex:MBITES-UsersGuide.Rnw:%
1 25 1 1 0 257 1 1 36 38 0 1 2 140 1 1 14 16 0 1 2 167 1 1 16 18 0 1 2 %
25 1 1 7 1 2 4 1 1 23 1 2 122 1 1 9 11 0 1 2 505 1}


\vspace{0.2in}

MBITES was designed and implemented by David L Smith, Sean Wu, and Hector Sanchez. MBITES extends previous modeling efforts developed by David L Smith, Amit Verma and Arnaud Le Menach. The GUI for MBITES was designed and implemented by Qian Zhang, Hector Sanchez, Sean Wu, and David L Smith. MBDETES, the differential equation simulation model developed as a bridging model for MBITES, was developed by John Henry, Sean Wu, and David L Smith. Anne Wilson and Steve Lindsay were the primary consultants for {\em Anopheles} and Tom Scott for {\em Aedes}. Important contributions were made by Daniel Citron, Alec Georgoff, and John Marshall. 

This document is a users guide to MBITES (Mosquito Bout-based and Individual-based Transmission Ecology Simulator). It is designed to give an oveview of the software for both developers and users. The best source of documentation for code developers is the code itself. Other documents describing MBITES include a manuscript for peer review and R documentation \footnote{Wu S, S{\' a}nchez-Castellanos H, Henry J, Zhang Q, Citron D, Verma A, Reiner RC Jr., Smith DL. Vector bionomics and vectorial capacity as emergent properties of mosquito behavior and ecology. For submission to {\em PLoS Computational Biology}.}. The code for MBITES is being developed for public release, when it will be published in a permanent archive. A GUI is available for configuring MBITES. It is distributed as part of the standard MBITES package. 

\clearpage 
\tableofcontents 
\clearpage 

\section{Introduction}

\subsection{What is MBITES?}

\begin{displayquote}
{\em \bf Mosquito Bout-based \& Individual-based \\Transmission Ecology Simulator}
\end{displayquote}

MBITES is a software package for simulating adult mosquito ecology and behavior. The purpose is to advance the science and analysis for problems related to mosquito-borne pathogen transmission and vector control. Conceptually, MBITES has several parts:
\begin{itemize} 
\item a {\bf landscape} describing resources used by adult mosquitoes; 
\item the algorithms describing adult behavior as a sequence of {\bf flight bouts}; 
\item a set of configurable {\bf options} for simulating other aspects of mosquito life-history and behavior; 
\item the {\bf aquatic ecology}; and 
\item {\bf vector control}. 
\end{itemize} 
The following sections describe various features of MBITES. Each section begins with a brief summary of the natural history, a description of the heuristic followed by implementation notes, relevant details, and a guide to the code.

\vspace{0.1in}

\verb2NOTE: Version 1.0 is being developed as R code. Future 2 \\
\verb2versions will retain the R interface, but the current 2 \\
\verb2plan is to migrate the simulation code to C++2

\subsection{What is MASH?}

\begin{displayquote}
{\em {\bf Modular Analysis and Simulation for Health}}
\end{displayquote}


\begin{displayquote}
{\em {\bf NOTE:} The following section gives some background on the basic concepts and design principles of MASH. We recommend reading it at some point, but feel free to skip it if you want to get on with learning MBITES.}
\end{displayquote}

MASH is a framework for simulating human health events and related phenomena (\eg pathogen transmission by mosquito populations). MBITES is a flexible module that runs the adult mosquito component of MASH. The terms ``component'' and ``module'' and ``flexible'' all have a specific meaning in MASH. The following section explains the terminology used in MBITES and MASH and the rationale for developing it.

MASH is a simulation modeling framework, not a simulation model. To put it another way, the problem MASH was trying to solve was how to build a plug-and-play simulator for human health, not how to simulate human health. The core design problem for MASH was how to build an API (application program interface) defining how various components of a model interact. The philosophy of MASH was that every component should have at least two modules: one would be highly realistic and simulate the relevant phenomena in exquisite detail. Another would be dead simple. 

The first disease to be implemented within MASH was malaria. The collection of all mechanistic models of human malaria have the same basic structural parts: a model of human malaria infection and immunity; a model of mosquito populations, including infection and development of malaria parasites; and some model of immature mosquito  populations in their aquatic habitats. In MASH, each one of these parts is called a \verb1COMPONENT1. Four distinct biological processes determine how these three components interact: malaria parasites can be passed from mosquitoes to infect humans when mosquitoes probe in anticipation of taking a blood meal; parasites infecting humans can be passed in the blood meal, infecting mosquitoes; adult mosquitoes lay eggs in aquatic habitats; and the eggs hatch, develop through four larval instars, then pupate and emerge as imago (the mature form) from the aquatic habitats to begin their lives as volant adults. A structure allowing MASH components to interact is called a \verb1QUEUE1. In its most general form, transmission describes the events when pathogens are transmitted, either from mosquitoes to humans or {\em vice versa}, but it also the dispersion of pathogens, the distances traveled by the pathogens between infection events. The \verb1LANDSCAPE1 is the part of the model that determines how dispersion is defined. 

To be truly plug-and-play, it must be possible to run a component in standalone mode, by passing the inputs from other components trivially as parameters. This is what we have done here. 

MBITES is the exquisitely detailed module for simulating the blood feeding behavior of adult mosquito populations. This document also describes how MBITES is coupled to the component describing the aquatic ecology. We describe how MBITES can be used to simulate pathogen transmission among populations of humans and mosquitoes; here we describe how that would happen without actually simulating transmission. This document describes how to run MBITES in standalone mode. 

The structure of MASH is summarized in the following way:
\begin{itemize}
\item MASH Structure
  \begin{itemize}
  \item A population of \verb1HUMANS1 and their health states and health events is the core of MASH. 
  \item A \verb1COMPONENT1 is a basic functional part of any human health model. 
  \item A \verb1LANDSCAPE1 defines the spatial scaffold for interactions. 
  \item A \verb1QUEUE1 is a component level structure allowing two distinct components to interact. 
  \end{itemize} 
\item A \verb1Module1 is a specific set of algorithms that instantiates a component. A module could be flexible in the sense that the same process could be represented by one of many different functional forms. 
\item A \verb1Synthetic Population1 is the fully configured set of initial conditions for the populations in a model.  
\end{itemize}
%
MASH is a framework for building models that is modular and flexible. This gives rise to a complementary vocaublary to describe the models that arise. We consider MBITES to be a CLASS of models: 
\begin{itemize}
\item Two models belong to the same ORDER if they belong to the same CLASS and share the same state space. 
\item Two models belong to the same FAMILY if they belong to the same ORDER and use the same functional forms.
\item Two models belong to the same GENUS if they belong to the same FAMILY and share the same parameters.
\item Two models belong to the same SPECIES if they belong to the same GENUS and share the same synthetic population.
\item An individual model is an individual member of a SPECIES defined by its random number seed. 
\end{itemize}
%
Species-level similarity is the default meaning of {\em \bf model} in MASH. Without some explicit way of mapping the state space, it is difficult to compare two models belonging to a different order. This hierarchical nomenclature system (borrowed from Linnaeus) makes it possible to talk about models defined by some other level of similarity (\eg model families).

\clearpage 

\subsection{What do I do with so many parameters?} 

\begin{displayquote}
{\em If you want to avoid a philosophical discussion of individual-based models, please skip ahead to} Section 1.
\end{displayquote}

Individual-based models have been criticized for many reasons. Most of these critiques are valid. William of Occam basically got it right, at least when it comes to problems related to statistical inference. Among many possible explanations, the simplest ones tend to be correct. With many parameters, you should be able to fit anything, and therefore know nothing! (There is even an old saying, attributed to various people: {\em Give me four parameters and I'll draw an elephant. Give me five and I'll make it wag it's tail.})  While this is a valid criticism of fitting individual based models to data, other problems arise when considering the future.

The past and the future are fundamentally different: there was one past, but there are many possible futures.  If the only thing you would ever wanted to do was predict the future, then your best bet would probably be to take the parsimonious model and run it forward. You probably won't get a better prediction, on average, from any single model. There are, however, other things you want to do, and other ways you migh use models to get a better prediction. For example, you might want to evaluate a policy, and here's where the old paradigm breaks down. Are the conclusions of your model robust to all the sources of heterogeneity the parsimonious model averaged over?

Consider a thought experiment. What if you could wave a wand and invent a model with an arbitrary degree of heterogeneity blindly fixing most of the parameters but leaving as many free parameters for fitting as you would have in an abstract and parametrically simple model. The rules of evidence would suggest the fits of these  models would be directly comparable. What gives any {\em a priori} weight to any particular model? There isn't any simple anser to that question, but none of the answers point automatically to the class of abstract and parametrically simple models most often used. A problem for science is that there are so many heterogeneous models, and they are very hard to specify. 

The problem of evaluating the robustness of some policy boils down to figuring out some way of waving a wand and fitting the very large set of heterogeneous models to the data in some systematic way, and then turning each model around to predict the future. The prediction made from the ensemble of all those fitted heterogeneous models should be more robust than the simple one. MBITES is an honest stab at doing {\em that}. 

How far can we get if we just ignore heterogeneity? The cheeky answer is that we can do everything and be assured we will never find a better answer, so long as we never go looking for one. If we're honest about the peformance of predictive models for infectious diseases, we'd have to admit we haven't been very successful. When the problem comes to giving advice, we don't know if we're doing any better.  Obviously, we don't have a sound basis for ignoring heterogeneity, however inconvenient it might be. That is, in part, why MBITES exists. 

Since we can't wave our wands, we need a way of dealing with the daunting problem of heterogeneity, besides the easily accepted answer; ignore it. MBITES has taken a different approach. The flexibility built into MBITES makes it comparatively easy to configure a landscape with any degree of heterogeneity, but it creates a new problem of how to systematically explore heterogeneity.  Our approach is to start from the same point as others -- we begin by anchoring the simulation model to the case that most closely resembles the homogenous approximation, and then we steadily add heterogeneity. There is a logical progression: 
\begin{itemize}
\item If it's not necessary, turn it off. 
\item If it's on, set everything to a constant value. 
\item If a constant value won't do, try a line. 
\item If the value isn't the same everywhere, see have far you can get with the simplest way of incorporating noise.
\item If there's a pattern, base it on a case study.  
\end{itemize}
To put it another way, we honor William of Occam by preferring simple answers, in part, so we can say whether something we've observed can be explained with a simpler model. Our question, though, is not to explain the past but to know with some confidence that we are making robust policy recommendtions for the future. With these principles in mind, we've developed a set of functions and utilities to set up landscapes that roughly follow these rules. 

\clearpage 

\section{Getting Started}

MBITES is a framework developed for individual-based simulation of adult female mosquito behavior. It was built for the purpose of understanding mosquito-borne pathogen dynamics and control, but it was developed around the flight bout as the irreducible or ``atomic'' unit of mosquito behavior. The name says what it is: a {\bf M}osquito {\bf B}out-based and {\bf I}ndividual-based {\bf T}ransmission {\bf E}cology {\bf S}imulator (MBITES). 

MBITES is not a model. It is a framework for building models. MBITES was developed with {\bf MODULARITY} and {\bf FLEXIBILITY} in mind. For our purposes, modularity describes a kind of built-in functionality that can be turned on or off, while flexibility describes a suite of built-in functional forms. (The distinction is meaningful in some deep sense, as described in section 1.2 above, but might not be so important for most end users.) With modularity and flexibility, there are many different options availble for building spatially explicit and highly realistic individual-based simulation models of mosquito behavior. These are linked to an equally large number of options for building models of human behavior and pathogen transmission dynamics. The built-in modularity and flexibility makes it possible to expand the number of options, so MBITES is like other programs. To build a model and simulate mosquito populations, it is necessary to configure the functions and parameters so that it looks like the specific mosquito species in its ecological context. This requires learning something about mosquitoes as well as something about models. 

\subsection{The Natural History of Mosquitoes}

MBITES is highly {\em mimetic} or {\em realistic}. It was designed to have a close correspondence between what mosquitoes do in the model and published descriptions. The female behavioral bouts are based on a basic description of adult female behavior. A newly emerged female mosquito imago is pre-gonotrophic; she must mate before she can produce eggs. Maturation could also require one or two blood or sugar meals. Sperm is stored in an organ called a spermatheca, so mating only needs to happen once, but depending on the species, it could occur repeatedly throughout life.  

After maturation, an adult female begins a cycle of blood feeding and egg laying. After a search to find a potential blood meal host, she approaches a host in an attempt to blood feed. After the bloodmeal, the mosquito typically rests to shed some of the water weight. Blood is provisioned into eggs that require some time to mature. Once mature, she initiates a search to find aquatic habitat and then lays eggs. Sugar feeding can provide energy for flight, metabolism, and other activities.

Each activity -- mating, searching, blood feeding, egg laying, or sugar feeding -- requires one or more flight bouts. In the model, the intent in each one of these flight bouts follows a logical sequence, albeit probabilistically, through eight distinct {\em behavioral} states (see below). MBITES represents each kind of female mosquito behavioral flight bout explicitly, and the structure of each bout type is different, and it is possible to describe what happens during each bout in virtually unlimited detail. 
 
The progression through these states is partly determined by variables describing a mosquito's state (\eg eggs must be mature before egg-laying occurs), but features of the mosquito biology constrain the sequence, and progression through the states can be somewhat random and dependent on available cues and resources. Transition probabilities for behavioral states can be configured to match site-specific features or species-specific behaviors. MBITES can also simulate male mosquitoes, which mate and blood feed.  

What is also important for MBITES is that all of these activities must occur where resources are available on a landscape. The landscape is a formal structure in MBITES, a set of points where resources are available. MBITES is a simulation framework, but it has utilities for visualizing and analyzing the resulting population dynamic processes.

\subsection{Running MBITES}

The philosophy of MBITES is that the process of building a simulation is as important as the output of the simulation. It's important for the MBITES jockey (\ie any user) to understand the to understand how a model will be used and its limitations. Just because a model looks to us like reality doesn't mean it is. We encourage all MBITES users to cultivate a healthy skepticism about any claim made. In fact, MBITES was developed to destroy dogma as much as to mimic reality.  Please use it with caution. 

We recognize at least two broadly different kinds of activities: academic projects and policy analysis. Academic projects are designed to build on the knowledge base, while policy analysis is generally designed to address a specific question about what to do that has arisn in some  particular context. Regardless of the purpose, MBITES is a complex bit of software. The point of this user's guide is to help the MBITES jockey understand how to use it. 

To learn MBITES, a simulation called \demo\; has been configured to illustrate the features of the model. \demo\; is run as the first option within the MBITES-GUI. The MBITES-GUI also has some built-in functionality to set up a simulation model; by default, it starts with a set of configured options for \minmod, a model that makes the simplest assumptions for each section. 

This document is structured around \demo\; and \minmod. The following sub-sections describe the way MBITES represents a mosquito and some notes about the basic structure of a {\bf bout}. The next section describes the spatial template for simulation, called the {\bf LANDSCAPE}. In the sections that follow, there is a brief discussion of the natural history of a phenomenon as a process, and a discussion of the algorithm for simulating that process. This is followed by a discussion of the built-in options for each algorithm, including the way it was implemented in \demo; and \minmod. Implementation notes sometimes follow, which take a deeper dive into the code, but code documentation is (by and large) built into the code. In some cases, the options discuss planned modifications to MBITES that have not been fully implemented in the current version.  

\subsection{The {\em in Silico} Mosquito}

The main tasks for MBITES were 1) how to represent a mosquito in a simulation model that could accurately mimic what mosquitoes do in real life; 2) how to link this code to a generalized pathogen simulation framework; and 3) how to build bridging models to elucidate the underlying processes relevant for pathogen transmission and mosquito fitness. To do all this in a computer, a mosquito must be abstracted down to a set of  of variables representing its state, including quantitative or categorial variables that track where a mosquito is, its physiological status, behavioral intent, and etc. The list of built-in variables that could be tracked by MBITES is the following: 
%
\begin{itemize}
\item A unique identifier for each mosquito, used for bookkeeping, pedigrees and inheritance.
\item Behavioral state.
\begin{itemize} 
  \item Does the mosquito intend to blood feed [B], lay eggs [O], mate [M], or sugar feed [S].
  \item Is the mosquito making an attempt, or is it searching for resources? NOTE: A single letter name usually refers to the attempt, and two letters indicate a search (e.g. Ms = searching for a mating site, and Ss is searching for sugar). In MBDETES, we also use the single letters for blood feeding search [F=Bs] and the egg laying search [L=Os].
  \item While not explicitly recognized as states, there is a significant resting period that follows the blood meal, called the post-prandial resting period.  Estivation can also occur in the model, but it is (usually) a seasonally induced, but very long resting period.   
\end{itemize}
\item Location. 
\begin{itemize} 
\item Where is the mosquito in space? (\ie at what site) 
\item Resting microsite. On what kind of surface  is the mosquito resting (\eg inside wall)?
\end{itemize}
\item Variables describing a mosquito's condition or status
\begin{itemize}
\item Energy reserve levels. How much sugar does a mosquito have in reserve? 
\item Physical condition. How much physical damage has a mosquito sustained? 
\item Physiological condition. How much physiological damage has a mosquito sustained?
\item Infectious status. Is a mosquito infected or infectious with a pathogen? 
\item Sexual maturity. Has a mosquito mated? 
\end{itemize}
\item Variables describing a mosquito's age
\begin{itemize}
\item Chronological age. How old is a mosquito? MBITES keeps track of the birth moment. 
\item Reproductive age. How many egg batches has a mosquito laid (\ie parity)? 
\end{itemize}
\item Type. 
\begin{itemize}
\item What is a mosquito's genotype / phenotype for insecticide resistance, genetic modification, and for dealing with evolutionary problems?
\item What is the type of the male mosquitoes with which a female has mated?
\end{itemize}
\end{itemize}
These states change as a mosquito ages and goes through its activities. 

\subsection{Behavioral States and Flight Bouts}

A crude description of a mosquito's activities would be that a mosquito's life is a sequence of flight bouts, and that each bout is an intentive flights to accomplish something. The behavioral states and state transitions are the key to understanding MBITES. A mosquito's intent determines what a mosquito does, and the intent changes as a mosquito accomplishes one thing and then moves to the next. MBITES mimics these intentive flight bouts, tracking a  mosquito from its launch to initiate one bout up to the point when it launches itself to initiate the next bout. Conceptually, the structure of a bout (generically denoted [{\bf X}]) is:  {\bf Launch} - {\bf Try} - {\bf Land} - {\bf Rest}. MBITES has  eight types of behavioral bouts: 
\begin{itemize}
\item[\Fbout] :: Blood feeding search  to find a place where blood feeding can occur
\item[\Bbout] :: Blood feeding attempt to approach a host for a blood meal and to blood feed
\item[\Lbout] :: Egg laying habitat search to find a place where egg laying can occur
\item[\Obout] :: Attempt to lay eggs / oviposit 
\item[\SSbout] :: Sugar feeding search bout (Optional)
\item[\Sbout] :: Sugar feeding attempt (Optional)
\item[\MSbout] :: Mating search bout (Optional)
\item[\Mbout] :: Mating attempt bout (Optional)
\end{itemize}
%
The final state of a mosquito is \Dead, for ``dead." The {\bf Try} part (\ie the intentive flight) of each bout type  is different -- what happens between launching and landing is elaborated upon in each bout. The intentive flight is assumed to be of comparatively short duration and (if successful) to lead to a state transition. 

\subsection{Landing, Resting, and Survival}

Aside from all the specific activities that occur during each bout, each bout executes a set of algorithms to simulate a set of common tasks to simulate changes to a mosquito's behavioral, physical and physiological states: 
\begin{itemize}
\item During a flight bout, a mosquito might change location. If a mosquito is searching for resources, the function \verb1rMove1 is called to determine where a mosquito ended up (see Search and Dispersion, below). 
\item Determine the microsite where a mosquito landed. After each bout, a mosquito lands at a particular microsite in each place on the landscape, determined by the site type and called by the function \verb1restingSpot1.
\item Determine energy usage. The mosquito's energetic state is then updated, with \verb1energetics1, which calls another function to determine whether the energy has been depleted to a point that the mosquito requires sugar. If so, then \verb1starved=TRUE1 is set. (described in the Section {\em Energetics of Flight, Blood Feeding, and Sugar Feeding}, below) 
\item Determine whether a mosquito survived the flight and the subsequent resting period. Even if a mosquito encountered the specific hazards associated with the bout, other mortality can occur as a result of flight stress with the function \verb1surviveFlight1. There is also the possibility of predation or other site specific hazards while resting, called with the function \verb1surviveResting1. (described in the Section {\em Survival}, below)
\item A function \verb1updateState1 is called to determine a mosquito's next behavioral state. A function called \verb1oogenesis1 is called to see whether a mosquito is gravid. The functions above set various switches to determine if a mosquito is alive, starved, gravid, or randy. If alive, a mosquito will go in search of blood, but this can be superceded by another uone of these other urges. If a mosquito is starved (meaning it has determined there is a need to seek a sugar meal), it will go in search of sugar regardless. If it is gravid, it will tend to change state, though a function can be configured for refeeding by mosquitoes, even if they are gravid (see the blood feeding discussion, below).  
\item Finally, the resting period determines the waiting time to launch the next bout, depending on its new behavioral state. Here, two conditions are checked: one for mating, and the other for estivation. Mating is triggered by the time of day (if a mosquito is not starved), and estivation is triggered by the time of the year. If mating is triggered as a behavioral state, a mosquito's behavioral state is changed to \Mbout\; or \MSbout. The waiting time is determined by a generic function \verb1timing1 that calls the function for that bout type (described in the Section  {\em Resting Times}, below).
\end{itemize}
These events can be different for each flight bout, depending on where a mosquito rests and what happened during the previous bout. The details of these activities matter for some kinds of malaria control.  There are, of course, other hazards that can be encountered that collectively make up vector control. 

Every mosquito event can be logged so it is possible to study mosquitoes using  \verb1historyTrack1, which outputs a mosquito's state at the end of each bout. Output is formatted as a JSON file, and most of the analysis is done post-processing. 


\subsection{Options}

Some bouts are required, and others are optional. The core bouts are \Fbout, \Bbout, \Lbout, and \Obout\; must all be in place to run MBITES at all, but \Sbout and \Mbout are optional states to be turned on and utilized, if it suits the problem at hand. \demo\; uses all the states. In \minmod, only the core bouts are used; the others must be turned on and configured.

\clearpage 

\section{Landscape}

MBITES organizes activities around a landscape structure, called \verb1MICRO1. The landscapes in MBITES are micro-epidemiological structures, representing the locations at a very fine grain, rather than the patch-based or metapopulation structures that are also commonly used to model malaria spatial dynamics. We think of a \verb1MICRO1 landscape as being comprised of different point sets called {\em sites} where the necessary resources can be found, and a set of \verb1QUEUES1 -- objects that are used to coordinate the exchange of information between different model components. 

\subsection{Sites}

A site is a location on the landscape were events occur. Conceptually, each site represents a small area, such as the area surrounding a household.

\paragraph{Mosquitoes} 
There are four types of point sets with the resources required for adult mosquito behaviors:
%
\begin{itemize}
\item[\fp] :: haunts or blood feeding sites where feeding occurs
\item[\lp] :: habitats, short for aquatic habitats where egg laying occurs
\item[\ssp] :: sugar, or sites where sugar feeding occurs
\item[\mm] :: mating sites
\end{itemize}
%
The purpose of naming sites is to limit the number of locations where activities can occur. In MBITES, blood feeding can only occur at a blood feeding site, which depends on the transmission system.  For some vectors, this will include all the dwellings where people live. Egg laying can only occur in suitable pools of water. Sugar feeding requires a flowering plant or exposed fruit. Mating sites are species specific places. 

%\paragraph{Points of Care} 
%Another set of points describes where humans go to seek care:
%
%\begin{itemize}
%\item[c] :: clinics or community health workers
%\end{itemize}
%
%While this is the way we think about landscapes, there is in fact one master point set where {\em any} events happen, and each unique point is called a {\em site}. A set of boolean switch describes the kinds of events (\eg blood feeding, egg laying, sugar feeding, mating, or care seeking) occurring there.  

Setting up a site is different from the simulation.  To set up a site, it is given a location in space ($x,y$) and a search weight ($\omega$). These are used to compute dispersal kernels, probability mass functions determining where a mosquito goes, depending on its location and state (see Location, Search, \& Dispersal). A site is also given a {\em site type}. When a mosquito lands, a function \verb1restingSpot1 is called that determines the microsite where a mosquito rests. 

\subsection{Queues}

The landscape is the spatial template upon which objects from different components of the model can interact. Each structure that allows two components of MASH to interact is called a \verb1QUEUE1.

Humans and mosquitoes interact mainly through blood feeding, but for this to occur, there must be a way for humans and mosquitoes to meet up on the landscape. Similarly, adult female mosquitoes lay eggs in aquatic habitats, and after hatching, developing through four laval instars, and pupating, mature adults emerge. The populations of immature mosquitoes are a different component, so egg laying and recruitment facilitate through queues makes modularity possible. Finally, another \verb1QUEUE1 allows male and female mosquitoes to find each other at swarming sites.  

\paragraph{atRiskQ} The \verb1atRiskQ1 holds information about all the blood hosts who are present at each blood feeding site. A paired function, called \verb1chooseHost()1 returns the identity of the host a mosquito will approach during the blood feeding attempt. Note that this queue is designed to be set up as a part of MASH when there is a dynamic host population.  MBITES has a static host population that is set up to make it possible to simulate mosquito populations in standalone mode. 

\paragraph{eggQ} Each aquatic habitat has an \verb1eggQ1 that olds all the eggs laid at each aquatic habitat. This queue facilitates modularity by making it possible for any adult female module to interact with any ecology module describing immature aquatic forms. Note that some module must be set up for mosquit population dynamics.  To run MBITES in standalone mode, a module called \verb1Emerge1 is called that has mosquitoes emerging, thereby ignoring egg laying. 

\paragraph{imagoQ} The imago is the volant mature adult insect. The \verb1imagoQ1 holds females emerging from aquatic habitats after pupation. This queue facilitates modularity by making it possible for any adult female module to interact with any ecology module describing immature aquatic forms. The numbers and types of mosquitoes emerging come from \verb1Emerge1.

\paragraph{matingQ} Adult male mosquitoes form swarms. The \verb1matingQ1 holds the identity of all the males in a swarm -- the female chooses a male to mate with. 

\clearpage 

\demo\; runs a simulation with all of these features enabled, while \minmod\; runs a substantially reduced simulation with just one point of each type. 

\subsection{Location, Search, \& Dispersion}

Each site in MBITES has a location ($x,y$) and a search weight ($\omega$) that are used to configure the function \verb1rMove1, which determines how mosquitoes disperse on the landscape. Each time a mosquito makes a behavioral state transition or gets frustrated in its attempts to land, it may change location. When a mosquito moves, the function \verb1rMove1 is called to determine its next location. This must be set up for each point in every point set where a mosquito may start a search to every point where a mosquito might end a search. This can be computationally intensive, so \verb1rMove1 sets up lookup tables. 

A flight search bout for blood feeding could begin anywhere. A probability mass function is called to determine which member of a destination point set it ends up at $D$ depending on its point of origin, $o$:
%
\begin{equation}
%
\left\{  o \right\}
\overset{{\cal K}_o}{\rightarrow}
\left\{  D \right\}
%
\end{equation}
%
The origin point may actually be in the destination point set, depending on the behavioral state.

\subsubsection{Options}

\ND

\subsubsection{Implementation Notes}

In the current version of MBITES, the dispersal kernel is pre-processed and the probabilities are stored at each site. This is hard wired into the structure of each site. 

\subsection{Site Types and Resting Spot}

Each time a mosquito lands, it must execute some code to determine the microsite where it lands. This has been implemented to make it possible to simulate some kinds of vector control and other mosquito behaviors (\eg house entering) in a highly mimetic way. Each sites has a {\em site type} that determines the landing microsites, which are used primarily for vector control. MBITES has two built-in site types. The simple site type and a homestead. 

\paragraph{[0] Simple Site Type}

The simple site type has two built-in options to stay or leave, if a mosquito is arriving or already here. The probability a mosquito does not stay after arriving from a search is already determined by the search parameter for a success. Landing behavior is set by a single parameter:
\begin{equation}
\begin{array}{r|c|c|}
& \mbox{stay} & \mbox{leave} \\ \hline  
\mbox{arriving} & 1 & 0 \\ \hline 
\mbox{here} & p & 1-p \\ \hline 
\end{array}
\end{equation}
If a mosquito leaves, then \verb1search=TRUE1 is set. 

\paragraph{[1] The Homestead}

A homestead is a site type built to accomodate some modes of vector control. The options are to land on an indoor wall, to land on an outdoor wall, to land outside on vegetation, or to leave without landing.  

\begin{equation}
\begin{array}{r|c|c|c|c|c|}
& \mbox{stay} & \mbox{vegetation} & \mbox{outside wall} & \mbox{inside wall} & \mbox{leave}  \\ \hline  
\mbox{arriving} & a_{11} & a_{12} & a_{13} & a_{14} & 0\\ \hline 
\mbox{vegetation} & a_{21} &  a_{22} & a_{23} & a_{24} & a_{25} \\ \hline 
\mbox{outside wall} & a_{31} &  a_{32} & a_{33} & a_{34} & a_{35} \\ \hline 
\mbox{inside wall} & a_{41} &  a_{42} & a_{43} & a_{44} & a_{45} \\ \hline 
\mbox{leave} & a_{51} &  a_{52} & a_{53} & a_{54} & a_{55} \\ \hline 
\end{array}
\end{equation}



\subsubsection{Options}

There are two built-in site types. In \demo, any site where blood feeding occurs is a Homestead; the rest use the Simple Site Type. In \minmod, all sites are simple. 

By default, the same matrix is used for all behavioral bouts, but it is also possible to configure different matrices for each behavioral bout: 
\begin{itemize}
\item \verb1homesteadRestingMatrix=01 sets a single matrix for all bouts
\item \verb1homesteadRestingMatrix=11 sets different matrices for each bout, each one needing to be configured. 
\end{itemize}

\subsubsection{Implementation Notes}

The function \verb1restingSpot1 determines whether a mosquito lands on a particular type of surface or whether it is repeatedly frustrated in attempts to land, thereby inducing a new search. An important distinction for homesteads is whether a mosquito enters a house, introducing the possibility of encountering an eve tube. Another important distinction is whether a mosquito lands on an interior wall that has been sprayed with pesticides. 

Finally, if a mosquito is frustrated in its attempts to land and chooses to leave, then \verb1search=TRUE1 is set.

\paragraph{R Code}

\begin{Schunk}
\begin{Sinput}
> mbites_restingSpot <- function(){
+   if(self$isActive()){
+     if(self$searchFail()){
+       private$lspot = "l"
+       private$search = TRUE
+     } else {
+       oldSpot = private$lspot
+       private$lspot = self$newSpot() # choose new lspot
+       if(oldSpot != "i" & private$lspot == "i"){
+         self$enterHouse() # enterHouse
+       }
+     }
+   }
+ }
\end{Sinput}
\end{Schunk}

\subsection{The Blood Meal Host Environment}

Each blood feeding site has an object that stores the blood meal hosts present, one of the \verb1QUEUE1 objects that stitches MASH together, called the \verb1atRiskQ1. To run MBITES as a standalone program, it is necessary to create a static \verb1atRiskQ1. In either case, the non-human hosts and mosquito traps at a site all attract mosquitoes to some degree. The \verb1atRiskQ1 holds the information about all potential bloodmeal hosts. Some of these hosts could potentially be turned into a kind of vector control: cattle can be fed ivermectin or sprayed with a pesticide. 

\subsubsection{Options}

\paragraph{Static Host Populations}

To run MBITES in stand alone mode, it is necessary to configure the host population environment.  The basic parameter $Q$ is used to tune the weights of alternative hosts at each site such that it sets {\em on average}, the proportion of blood meals taken on a human.

\subsubsection{Implementation Notes}

\subsection{Site-specific Hazards}

Each site has a number that is the probability of surviving the local hazards. By default, this is set to $0$, signifying there is no mortality, but if there is some site-specific mortality that occurs (\eg from predation) this can be set to some positive number between $0$ and $1$ representing the probability of dying from a site-specific hazard that is not otherwise explicitly represented in the model. These site-specific hazards operate in addition to any mortality occuring as a result of vector-control, which is configured separately. 

\subsubsection{Options}

\paragraph{Configuration}

\begin{itemize}
\item \verb1landingHazard_mean=01
\item \verb1landingHazard_pdf=NULL1
\item \verb1flightHazard_mean=01
\item \verb1flightHazard_pdf=NULL1
\end{itemize}

MBITES provides some built-in utitities for setting landscape options, as described in the sections below. It is also possible to simulate some behaviors on a minimal landscape, or to set up landscapes and import them into MBITES (see MICRO I/O )

\subsection{Making Landscapes}

There are so many ways to build a landsdcape that we have not gone beyond a simple set of algorithms in MBITES. We presume users will want to provide their own point sets, so we provide IO. MBITES expects a table of this format for each resource type: 

\begin{itemize}
\item [x] :: east-west coordinate
\item [y] :: north-south coordinate
\item [S] :: the site type
\item [w] :: the search weight
\item [h] :: the site-specific hazard
\item [z] :: the search weight for selecting non-human hosts. 
\end{itemize} 

\paragraph{I/O for Aquatic Habitats}

\begin{itemize}
\item [x] :: east-west coordinate
\item [y] :: north-south coordinate
\item [S] :: the site type
\item [w] :: the search weight
\item [h] :: the site-specific hazard
\item [H] :: the habitat type
\item [...] :: parameters for the habitat 
\end{itemize} 

\paragraph{I/O for Sugar Sources}

\begin{itemize}
\item [x] :: east-west coordinate
\item [y] :: north-south coordinate
\item [S] :: the site type
\item [w] :: the search weight
\item [h] :: the site-specific hazard
\end{itemize} 

\paragraph{I/O for Mating Sites}

\begin{itemize}
\item [x] :: east-west coordinate
\item [y] :: north-south coordinate
\item [S] :: the site type
\item [w] :: the search weight
\item [h] :: the site-specific hazard
\end{itemize} 

% \begin{itemize}
% \item [x] :: east-west coordinate
% \item [y] :: north-south coordinate
% \item [S] :: the site type
% \item [h] :: the local resting hazard (\ie probability of dying while resting here)
% \item [f] :: the blood feeding search weight (0, if not in $\left\{f\right\}$)
% \item [l] :: the number of aquatic habitats (0, if not in $\left\{l\right\}$)
% \item [s] :: the number of sugar sources (0, if not in $\left\{s\right\}$)
% \item [m] :: the mating search weight (0, if not in $\left\{m\right\}$)
% \item [z] :: the search weight for selecting non-human hosts. 
% \end{itemize}
\clearpage 

\section{Resting Times}

After landing, how long does it take a mosquito to launch again? This is the question addressed by the set of functions called \verb1timing1. Here, the waiting time to the next bout varies by the mosquito's behavioral state, and two additional checks are made to see if a mosquito will estivate or initiate a mating bout. 

\subsection{Options}

By default, all of these waiting times are exponentially distributed. The alternative is gamma distributed waiting times. 

\paragraph{Parsing}

For the behavioral states F,B,R,L,O, \& S, these are the timing options: 
%
\begin{itemize}
\item [{\bf 0}] \verb1boutX_tte_pdf=exponential1

  $$min + \mbox{rexp}(.,1/(mean-min))$$
  \begin{itemize}
  \item \verb1boutX_tte_mean1
  \item \verb1boutX_tte_min=01
%  \item \verb1boutX_tte_diurnal1
%    \begin{itemize}
%    \item [0] = FALSE 
%    \item [1] = TRUE
%      $$min + \mbox{rdiurnal}(.(mean-min), t, peak)$$
%      \begin{itemize} 
%      \item \verb1boutX_tte_peak=01 (midnight)
%      \item \verb1boutX_tte_amp=11 
%      \end{itemize}
%    \end{itemize}
  \end{itemize}
\item [1] \verb1boutX_tte_pdf=gamma1
  $$min + \mbox{rgamma}(.,mean-min, scale)$$
  \begin{itemize}
    \item \verb1boutX_tte_mean1
    \item \verb1boutX_tte_scale1
    \item \verb1boutX_tte_min=01
  \end{itemize}
  
\end{itemize}

\clearpage 

\section{Survival}

How and where do mosquitoes die? Since no one really knows, MBITES makes it possible for mosquitoes to die every which way. The wide variety of options for death in MBITES creates a conundrum that can't be ignored in models that don't look at mortality purely phenomenologically. In MBITES, it is necessary to make specific assumptions that determine how, where, and when mosquitoes die. 

The default setting for mortality in MBITES is for all mosquito mortality to occur as a result of flight stress. The basic probability of dying is a bout-specific probability of surviving the stress of each flight, which is configured differently for each bout. 
%
\begin{itemize}
%
\item \verb1boutX_surviveFlight1
%
\end{itemize}
%
Surviving flight stress can be modified in several ways that add mortality: senescence increases mosquito death rates as they age; with flight energetics, mosquitoes die as they run out of energy; mosquitoes can accumulate damage; and excess mortality can occur as a result of the blood meal. Descriptions of how to configure these options are found below.

MBITES can also be configured to have mosquitoes die from site-specific hazards associated with flight or landing. By default, all mortality from hazards are associated with landing. To set the shift some mortality away from flight stress and towards landing hazards, or from landing hazards to flight hazards, the following parameters are set:  
%
\begin{itemize}
%
\item \verb1mortalityFromHazards=01 The parameter value is a proportion, so it is in the interval $[0,1]$. If it is set to a number other than zero, then it determines the geometric split for survival among flight stress and hazards (\ie if $s_0$ and $s_1$ were survival from two different mechanisms, then $s_0 = s^q$ and $s_1= s^{1-q}$ such that $s = s_0 s_1$. This is what we mean when we say that $q$ determines geometrically the weight on $s_0$). That value then sets the {\em arithmetic} average for the total local hazards. 
\item \verb1propHazardsFlight=01 By default, local hazards are all landing hazards.  Changing this parameter to some number in the interval $[0,1]$ shifts that proportion of the hazards to flight hazards, not landing hazards. 
%
\end{itemize}
%
These landing hazards can be distributed heterogeneously. The distribution is configured as part of the landscape. Two sliders in the MBITES-GUI make it possible toset parameters to affect how much mortality occurs as a result of flight stress compared with hazards. 

The values given are set under a specific set of assumptions about the landscape configuration. Feedbacks between mortality and the frequency with which certain sites are visited can change the proportion of mortality occurring at each site. For this reason, we would like to emphasize that configuring MBITES to do something requires careful thought. We recommend running diagnostics after setting parameter values or changing some aspect of the landscape.  Even then, a simulation model can behave differently than expected (\eg because of population dynamic feedbacks). To put it another way, the MBITES-GUI is designed to help calibrate the model, but there will always be a need for analysis. 

\subsection{Flight Stress}


\subsubsection{Implementation Notes} 

When \verb1surviveFlight()1 is called, the function \verb1get_surviveFlightProb()1 returns the
probability of surviving, and a random number is drawn
to determine whether the mosquito survived the stress associated
with flight. The function is called from {\bf
MBITES-GenericSurvival.R} and at the moment, it looks like this: 
%
\begin{Schunk}
\begin{Sinput}
> mbitesGeneric_surviveFlight <- function(){
+   if(self$isActive()){
+     p = self$get_surviveFlightProb()
+     if(private$FemalePopPointer$get_MBITES_PAR("TATTER")){
+       private$damage = private$damage + self$WingTattering()
+       p = p * self$pTatter()
+     }
+     if(private$FemalePopPointer$get_MBITES_PAR("SENESCE")){
+       p = p * self$pSenesce()
+     }
+     if(runif(1) < 1-p){
+       private$stateNew = "D"
+     }
+   }
+ }
\end{Sinput}
\end{Schunk}


 
\subsection{Senescence}

\begin{itemize}
\item [0] \verb1senesce_function=NULL1
\item [1] \verb1senesce_function="gompertz"1
  \begin{itemize}
  \item \verb8senesce_gompertz_p18
  \item \verb8senesce_gompertz_p28
  \end{itemize}
\end{itemize}

Senescence is defined in this model as a reduction in the
probability of surviving that is associated with a  mosquito's
chronological age, $a$. In MBITES, senescence has been
implemented as an increased probability of mortality, per bite,
that increases with the chronological age of the mosquito when it
takes a flight bout. The function describing
the probability of surviving is 
%
$$\frac{2+sns.b}{1+sns.b} -
\frac{e^{a*sns.a}}{sns.b + e^{a*sns.a}}.  
$$ 
%
\includegraphics{MBITES-UsersGuide-003}

{\bf NOTE: overall survival vs. age is difficult to compute without simulation. 
It would be good for the GUI to show surival of ~1,000 mosquitoes with and without senescence, 
and to compute (and show) the average age. Something like this:}  

\includegraphics{MBITES-UsersGuide-004}

\subsection{Physical and Physiological Damage}

The variable \verb1damage1 tracks all kinds of damage, including physical damage and contact with pesticides. Damage is cumulative,  and it is tracked on a scale of $[0,1)$. Two function families are required. One to simulate the amount of damage caused, and another to simulate the effects of damage on survival. Survival is assumed to be a mononic decreasing function of the amount of damage: 

\begin{itemize}
\item [0] \verb1damage_function=NULL1
\item [1] \verb1surviveDamage_function="zzSigmoid"1
  \begin{itemize}
  \item \verb8damage_zzSigmoid_zero8
  \item \verb8damage_zzSigmoid_p508
  \item \verb8damage_zzSigmoid_slope8
  \end{itemize}
\end{itemize}

\subsubsection{Wing Tattering}

Tattering describes physical damage resulting from flight. The amount of damage caused by each flight is simulated with a zero-inflated  $\beta$ PDF: 
%
\begin{itemize}
\item [0] \verb1tatter_function=NULL1
\item [1] \verb1tatter_function="zibeta"1
  \begin{itemize}
  \item \verb8tatter_zibeta_p08
  \item \verb8tatter_zibeta_mean8
  \item \verb8tatter_zibeta_ss8
  \end{itemize}
\end{itemize}

\subsubsection{Physiological Damage}

Physiological damage 

\clearpage 

\section{Search Bouts}

\section{Blood Feeding}

Blood feeding is modeled with two kinds of bouts: a search bout, in which a mosquito moves into the neighborhood of blood meal hosts; and an attempt bout, in which a mosquito approaches a host. The at

\subsection{Blood Feeding Attempt Bout [B]}

A blood feeding bout begins when a mosquito chooses a host to
approach. Information about the human hosts present in each haunt
are stored in the queue, called the \verb1atRiskQ1. Other potential blood
meal hosts (\eg cattle) can also be present (configured as part of the landscape). 
Each host has a weight, which is used in a multinomial probability distribution
function to select one of the hosts to approach. A function called \verb1chooseHost1 is called to select a host to approach.

After selecting a host, different functions determine the outcome
of an attempt, depending on whether the host is human or another
vertebrate animal. As a mosquito approaches a human attempting to
blood feed, there are several steps: an approach, probing, and
the blood meal. The mosquito could be frustrated or killed at
each one of these steps (\eg from swatting). A mosquito blood
feeding on a human becomes infected with a pathogen with some
probability. Pathogens mature by a (possibly temperature
dependent) rule determining EIP at some point in the future, if
the mosquito is still alive.

The process for feeding on a non-human vertebrate animals is
simpler; parameters determine the probability of death or success
(and their complement, failure followed by a new attempt).

After successfully blood feeding, a mosquito's behavioral state
changes to the post-prandial resting bout. If a mosquito survives
the attempt but does not succeed in taking a blood meal, then it
can either repeat a blood feeding attempt or leave the area to
search for a new site.

If a mosquito has successfully blood fed, a random variate is
drawn to determine the size of the blood meal, and there is
another function that determines whether the mosquito survives
the stress associated with blood feeding, depending on the size
of the blood meal.

\subsection{Blood Meal Size}

The size of a blood meal can be simulated explicitly, 
using built-in functions. 

\begin{itemize}
\item \verb1eggBatch_function=NULL1
  \begin{itemize}
  \item \verb1eggBatchSize1
  \item \verb1refeed_function=NULL1
    \begin{itemize}
    \item \verb1pr_Refeed1
    \end{itemize}
  \end{itemize}
\end{itemize}

\subsubsection{Refeeding}

\subsubsection{Overfeeding}


\paragraph{[B] Configuration}

default for blood meal size. 

\begin{itemize}
\item \verb1boutB_surviveFlight=0.991
\item \verb1chooseHost()1
\item \verb1bloodMealSize_function=NULL1
\end{itemize}

\paragraph{Human Blood Meal}


\begin{itemize}
\item \verb1boutB_surviveHumanApproach1
\item \verb1boutB_successHumanApproach1
\item \verb1boutB_surviveHumanProbing1
\item \verb1boutB_successHumanProbing1
\item \verb1boutB_surviveHumanBloodFeed1
\item \verb1boutB_successHumanBloodFeed1
\end{itemize}

\paragraph{Non-Human Blood Meal}

\begin{itemize}
\item \verb1boutB_surviveOther1
\item \verb1boutB_successOther1
\end{itemize}

\subsection{Pathogen Transmission}

Pathogens are transmitted from the mosquito to humans during probing. Pathogens are transmitted from humans to mosquitoes in the blood meal. The function \verb1probeHost1 is called whenever a mosquito successfully probes its host, and this function has functionality to infect a host, if the mosquito is infectious. The function \verb1bloodFeed1 is called whenever a mosquito successfully takes a blood meal, which has the functionality to infect a mosquito, if the blood meal host is infectious.  

\section{Oogenesis, and Egg Maturation}

The point of the oogenesis function is to determine the algorithm for simulating how blood is provisioned into eggs. The endpoint of oogenesis is to set \verb1gravid=TRUE1 if eggs are ready to be laid, or \verb1gravid=FALSE1 otherwise.

The sections on blood feeding, above, describe the mosquito behaviors and outcomes as a mosquito approaches a blood meal host. Pathogen transmission is also described in a separate subsection above. The purpose of this section is to describe all the events that follow a blood meal, and the options that can be implemented to simulate the blood meal and oogenesis (\ie egg making). These are all dealt with under the post-prandial resting bout, which is also described in detail here. 

What is relevant for the feeding cycle and for the behavioral transitions is provisioning of the blood meal into eggs and their maturation. How long does it take for eggs to mature and how does this affect a female mosquito's decision whether to blood feed again or lay eggs? One problem for mosquitos is that the blood meal is heavy. The protein in blood comes with a lot of water that makes it difficult to fly.  Losing this water takes some time. A second problem is that the eggs take some time to mature. How long does it take, and what does a mosquito do in the meantime? 



\subsection{Post Prandial Resting}

The outcome of a post prandial resting bout is either death,
another attempt to blood feed, or an egg laying search bout. 
In the simplest case, a single parameter describes the 
proportion of times a mosquito returns to feed after taking 
a blood meal. Other options deal with explicitly modeling the 
blood meal size and egg maturation (see below). 

If a mosquito survives, but the eggs are not mature or if she
decides to top-up, she will attempt another blood meal. Even if
eggs are mature, the mosquito may attempt to blood feed again,
depending on the size of the last bloodmeal. Otherwise, a
surviving mosquito makes an attempt to lays eggs.


\subsection{Egg Batch Size and Maturation}

Transition to egg laying behaviors depends on a rule determining
how eggs mature, depending on the time required for egg
maturation {\em vs.} the time required for a feeding bout. If egg
maturation occurs every feeding bout, then blood is provisioned
into eggs; the larger the blood meal, the larger the egg batch. A
mosquito could choose to feed again, depending on the size of the
blood meal / egg batch. Another option is that the eggs require
time or blood resources to mature. If eggs are maturing, a
mosquito repeats a blood feeding attempt until the eggs are
mature.

\begin{itemize}
\item \verb1eggBatch_function="proportional"1
  \begin{itemize}
  \item \verb1bloodMealSize_function="beta"1
    \begin{itemize}
    \item \verb1bloodMealSize_mean1
    \item \verb1bloodMealSize_ss1
    \item \verb1maxEggBatch1
  \end{itemize}
  \item \verb3refeed_function="1-zzSigmoid"3
    \begin{itemize}
    \item \verb1refeed_pzero1
    \item \verb1refeed_p501
    \item \verb1refeed_slope1
    \end{itemize}
  \end{itemize}
  \item \verb1eggBatch_function = "Aedes"1
    \begin{itemize}
    \item \verb1bloodMealSize_function="beta"1
    \begin{itemize}
    \item \verb1bloodMealSize_mean1
    \item \verb1bloodMealSize_ss1
    \item \verb1maxEggBatch1
    \end{itemize}
     \item \verb3refeed_function="fullBatch"3
      \begin{itemize}
      \item \verb1blood4FullBatch=x1 :: The number of full bloodmeals (or equivalent, usually $x >1$), required for a full batch of eggs. 
      \end{itemize}
  \end{itemize}
\end{itemize}

\paragraph{Configuration Options}

\begin{itemize}
\item \verb1boutR_surviveFlight1
\item \verb1eggBatch_function=NULL1
  \begin{itemize}
  \item \verb1eggBatchSize1
  \item \verb1refeed_function=NULL1
    \begin{itemize}
    \item \verb1pr_Refeed1
    \end{itemize}
  \end{itemize}
  \item \verb1eggBatch_function!=NULL1 (Section 5, below)
\end{itemize}

\section{Egg Laying}

\subsection{Egg Laying Attempt Bout [O]}

An egg laying bout involves a series of egg laying attempts where
a mosquito dies, successfully lays some fraction of the eggs and
initiates a new egg laying search bout, or is frustrated. If
frustrated, she will either either make another attempt to lay
eggs or initiate a new attempt somewhere else.

\begin{itemize}
\item \verb1boutO_surviveFlight1
\item \verb1boutO_success1
\item \verb1boutO_skip=NULL1
\item \verb1boutO_skip="even"1
  \begin{itemize}
  \item \verb1skip_N1 
  \end{itemize}
\end{itemize}



\section{Energetics and Sugar Feeding}

\subsection{Flight Energetics}

A variable track's the level of each mosquito's energy reserves,
and each bout depletes a mosquito's energy reserves by some
amount. These energy reserves could be restored by blood feeding,
depending on the species (a parameter is set to determine how
much energy a female mosquito is able to derive from blood).
Otherwise, a mosquito must restore its energy reserves by sugar
feeding. Sugar sources are distributed across the landscape, at
the points in $\left\{ s\right\} $, including possibly at
swarming sites, haunts, and habitats. Sugar feeding could occur
in two ways, through a sugar feeding bout or opportunistically.

The variable \verb1energy1 tracks a mosquito's energy reserves on a scale of $(0,1]$. A mosquito is dead if it ever has $0$ energy units. Basic energy use is configured by describing what proportion of energy is used during a flight. For convenience, this is the inverse of the total number of flights a mosquito could take if it started with a full load and never refueled, $EU$.  For example, after topping up to full and taking $n$ flights, a mosquito's energy would have the value $1-n/EU$. 

\begin{itemize}
\item \verb1flightEnergetics=FALSE1
\item \verb1flightEnergetics=TRUE1
  \begin{itemize}
  \item \verb1flightEnergetics_EU1 :: Energy use per flight, a number between $0$ and $1$. The inverse of this number is approximately the number of flights a mosquito can take before it runs out of energy (see above). 
  \item \verb8flightEnergetics_bloodTopUp_EU=08 :: Setting it to $x>0$ means a mosquito is able to take $x$ flights as a result of a full blood meal. 
  \item \verb1flightEnergetics_survival=NULL1 :: The mosquito dies only if its energy state falls below.
  \item \verb1flightEnergetics_survival="zzSigmoid"1 :: Survival declines to zero as the energy state approaches zero. 
    \begin{itemize}
    \item \verb8damage_zzSigmoid_zero8
    \item \verb8damage_zzSigmoid_p508
    \item \verb8damage_zzSigmoid_slope8
    \end{itemize}
  \end{itemize}
\end{itemize}

\subsection{Sugar Feeding}

Sugar feeding builds on flight energetics. 

\begin{itemize}
\item \verb1sugarFeeding = FALSE1
\item \verb1sugarFeeding = TRUE1
  \begin{itemize}
  \item \verb8flightEnergetics_sugarTopUp_EU=08 :: Setting it to $x>0$ means a mosquito is able to take $x$ flights as a result of a full blood meal.
  \item \verb1sugarFeeding_boutS=FALSE1
  \item \verb1sugarFeeding_boutS=TRUE1 (see {\em Sugar Feeding Bout}, below)
  \item \verb1sugarFeeding_opportunistic=FALSE1
  \item \verb1sugarFeeding_opportunistic=TRUE1 (see {\em Opportunistic  Sugar Feeding}, below)
  \end{itemize}
\end{itemize}

\subsubsection{[S] Sugar Feeding Bout}

A female mosquito's energy reserves are checked at the end of
every other kind of bout, and a function determines whether a
mosquito switches from some other state into an active sugar
feeding behavioral state, thereby initiating a sugar feeding
bout. A mosquito sugar feeding bout could begin at any haunt or
at another habitat, but it always ends in a sugar feeding site:
%
\begin{equation}
%
\left\{ f\right\} \cup \left\{ l\right\}  \cup \left\{ m\right\}
\cup \left\{ s\right\}  \overset{L}{\rightarrow} \left\{
s\right\}
%
\end{equation}
%
The outcome of a sugar bout is either death, a sugar meal, or
another attempt. If a sugar feeding bout succeeds, the sugar
reserves are topped up.

\begin{itemize}
\item \verb1sugarFeedingBout=TRUE1
  \begin{itemize}
  \item \verb1boutS_survive1
  \item \verb1initSugarFeedingBout_function = "zzSigmoid"1
    \begin{itemize}
    \item \verb1initSugarFeedingBout_pzero1
    \item \verb1initSugarFeedingBout_p501
    \item \verb1initSugarFeedingBout_slope1
    \end{itemize}
  \end{itemize}
\end{itemize}


\subsubsection{Opportunistic Sugar Feeding}

During any other kind of bout, opportunistic sugar feeding can be
triggered in a female if sugar is present. The functions
triggering obligate and opportunistic feeding are, by default,
shifted so that a female mosquito is more likely to feed
opportunistically than to actively sugar feed. The sugar reserves
of a mosquito are checked. The sugar at the source has a ``search
weight,'' and functions determine whether a mosquito chooses to
feed based on energy reserve levels and the weight. Another
function determines how much the reserves are topped up. If a
mosquito does sugar feed opportunistically, the mosquito could
die during the attempt. If it survives opportunistic feeding, its
behavioral state remains the same.

Male mosquitoes are obligate sugar feeders when they are not
mating. Sugar feeding will occur repeatedly until the time window
when mating occurs, when its state shifts back to mating.

\paragraph{Configuration}




\begin{itemize}
\item \verb1sugarFeeding_opportunistic=TRUE1
  \begin{itemize}
  \item \verb1grabSugar_function="zzSigmoid"1
    \begin{itemize}
    \item \verb1grabSugar_pzero1
    \item \verb1grabSugar_p501
    \item \verb1grabSugar_slope1
    \end{itemize}
  \end{itemize}
\end{itemize}

\clearpage 

\section{Mating Bouts and Maturation}

Mating requires simulating male mosquito populations. MBITES can
simulate both female and male individuals. After an adult female
has emerged from her pupal case and hardened, she is considered
to be {\em pre-gonotrophic}. The conditions for maturation
include a mating requirement and (possibly) an energetic
requirement, meaning that she could not lay eggs until she had
mated and satisfied an early life energy intake requirement. The
model has not pre-determined what these pre-gonotrophic energetic
requirements are, but provides a flexible set of functions and
parameters to stipulate, depending on the species and either
evidence or expert opinion. Mating and maturation can also be
turned off, such that a female is mature and mated upon
emergence.

Male mosquitoes emerge from aquatic habitats and by default
attempt to sugar feed (see below). At a specific time of each
day, the male behavioral state switches to {\bf M}, and it moves
to a mating point, $\left\{ m \right\}$, to be present for any
female arriving at that point to mate.
Male search thus always begins at a mating or sugar feeding site,
$\left\{ s \right\} $ to arrive at a mating site $\left\{ m
\right\}$:
%
\begin{equation}
%
\left\{ s\right\} \cup \left\{ m\right\}  \overset{{\cal
K}_M}{\rightarrow} \left\{ m\right\}
%
\end{equation}
%
If the male mosquito survives the flight, it enters a swarm. The
{\em swarming queue} tracks all the male mosquitoes present at
each point in $\left\{ m\right\} $ on each day. After the swarm
disperses, surviving males behavioral states switch back to {\bf
S} to sugar feed (see below).

A pre-gonotrophic female does not lay eggs, though she can blood
feed. Female mating bouts in this model occur only when she is in
a pre-gonotrophic state, which (like males) can be triggered
during a specific time window each day, when her state switches
to {\bf M}. The female search kernel could thus begins at any
mating, blood feeding, or sugar feeding site but always ends at a
mating site $\left\{ m\right\} $:
%
\begin{equation}
%
\left\{ f\right\} \cup \left\{ s\right\} \cup \left\{ m\right\}
\overset{{\cal K}_M}{\rightarrow} \left\{ m\right\}
%
\end{equation}
%
A female chooses from among the males in the queue at that
location. The probability of success is a function of the number
of males present in the swarm. When a female mosquito mates, the
identity of the male mosquitoes she mated with are chosen from
those present at the swarming site and stored. The outcome of a
mating bout is either death, success, or failure. After a mating
bout, a female's behavioral state switches to search for a blood
meal, {\bf F}.

An alternative to swarming behavior is opportunistic mating
during some other activity (\eg blood feeding). If males are
present where females blood feed, a random variate is drawn to
see if the blood feeding attempt bout is interrupted and mating
occurs. In such cases, ordinary hazards apply, and if the female
survives, her behavioral state does not change. (NOTE: we need to
add this possibility, just as we do for sugar feeding.)

\begin{itemize}
\item \verb1Mating = FALSE1
\item \verb1Mating = TRUE1
  \begin{itemize}
  \item \verb1maleMosquito="kernel"1
      \begin{itemize}
      \item \verb1maleDeathRate1 
      \item \verb1matingKernel1
      \end{itemize}
  \item \verb1maleMosquito="ibm"1
     \begin{itemize}
     \item \verb1maleFlightEnergetics1 : mirrors \verb1flightEnergetics1 for females 
     \item \verb1maleSugarFeeding=FALSE1
     \item \verb1maleSugarFeeding=TRUE1 :: mirrors \verb1sugarFeeding1 for females
     \item \verb1swarming=TRUE1 
%     \item opportunistic
%        \begin{itemize}
%        \item males
%        \item females
%        \end{itemize}
     \end{itemize}
  \end{itemize}
\end{itemize}

\subsection{[M] Mating Bout}

\begin{itemize}
\item \verb1swarming=FALSE1
\item \verb1swarming=TRUE1
  \begin{itemize}
     \item \verb1boutM_tte_function="matingTrigger"1 
        \begin{itemize}
         \item \verb1timeToSwarm1 
         \end{itemize}
     \item \verb1boutM=TRUE1 :: see below
       \begin{itemize}
       \item \verb1boutM_survive1
       \item \verb1boutM_success1
       \end{itemize}
     \item \verb1maleBoutM=TRUE1
        \begin{itemize}
          \item \verb1maleBoutM_survive1
          \item \verb1maleBoutM_success1
        \end{itemize}
  \end{itemize}
\item \verb1opportunisticMating=FALSE1
\end{itemize}

To do: \verb1opportunisticMating="Aedes"1, following Tom's description of males approaching females at blood feeding sites as they try to feed. 

\begin{itemize}
\item \verb1bornMature=TRUE1 
  \begin{itemize}
  \item \verb1Mating=FALSE1   
  \end{itemize}
\item \verb1bornMature=FALSE1 
  \begin{itemize}
  \item \verb1maturationEU=01 : Set to some other value to require energy.
  \item \verb1Mating=TRUE1 (see Options below)
  \end{itemize}
\end{itemize}

\clearpage 
\subsection{Male Mosquito Populations}

\clearpage 
\section{Estivation [E]}

Estivation is a state of inactivity. In this model, estivation is
induced seasonally, whenever certain conditions are met (\eg no
rain). If a mosquito survives estivation, it initiates a blood
feeding search bout at some point in the future.

\clearpage 

\section{Phenotypes, Inheritance and Evolution}

\planned 

\clearpage 
\section{Aquatic Ecology}

\subsection{Emerge}

\subsection{EL4P}

\clearpage 
\section{The MBITES GUI}

\clearpage 
\section{Analysis}

\end{document}
