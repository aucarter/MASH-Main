\documentclass{article}
\usepackage{csquotes}
\renewcommand{\baselinestretch}{1.2} 
\newcommand{\eg}{{\em e.g., }}

\usepackage{Sweave}
\begin{document}
\Sconcordance{concordance:MBITES-Setup.tex:MBITES-Setup.Rnw:%
1 3 1 1 0 126 1}


\begin{centering}
\huge{Setting up MBITES}\\
\vspace{0.3in}
\Large{Mosquito Bout-based and Individual-based \\ Transmission Ecology Simulator}\\

\vspace{0.3in}
\large{MBITES Development Team: \\ Sean Wu, Hector Sanchez, Qian Zhang, John Henry, Daniel Citron, Amit Verma, Arnaud Le Menach, David L Smith\\}

\end{centering}

\vspace{0.3in}

{\bf NOTE:} This document is designed to teach people how to configure the behavioral bouts and options for MBITES. Whereas other documents explaining MBITES describe the code and implementation, this document involves more vector biology. It assumes a basic familiarity with MBITES, either from {\em A User's Guide to MBITES}, or the manuscript being developed into a peer-reviewed journal article {\em Vector bionomics and vectorial capacity as emergent properties of mosquito behavior and ecology}

\clearpage 

\begin{displayquote}
{\bf \em If you want to avoid a philosophical discussion of individual-based models, please skip ahead to section 1.}
\end{displayquote}
\paragraph{Introduction} 

\begin{displayquote}
{\bf \em What do I do with so many parameters?}
\end{displayquote}

Individual-based models have been criticized for many reasons. Most of these critiques are valid. William of Occam basically got it right, at least when it comes to problems related to statistical inference. Among many possible explanations, the simplest ones tend to be correct. With many parameters, you should be able to fit anything, and therefore know nothing! (There is even an old saying, attributed to various people: {\em Give me four parameters and I'll draw an elephant. Give me five and I'll make it wag it's tail.})  While this is a valid criticism of fitting individual based models to data, other problems arise when considering the future.

The past and the future are fundamentally different: there was one past, but there are many possible futures.  If the only thing you would ever wanted to do was predict the future, then your best bet would probably be to take the parsimonious model and run it forward. You probably won't get a better prediction, on average, from any single model. There are, however, other things you want to do, and other ways you migh use models to get a better prediction. For example, you might want to evaluate a policy, and here's where the old paradigm breaks down. Are the conclusions of your model robust to all the sources of heterogeneity the parsimonious model averaged over?

Consider a thought experiment. What if you could wave a wand and invent a model with an arbitrary degree of heterogeneity blindly fixing most of the parameters but leaving as many free parameters for fitting as you would have in an abstract and parametrically simple model. The rules of evidence would suggest the fits of these  models would be directly comparable. What gives any {\em a priori} weight to any particular model? There isn't any simple anser to that question, but none of the answers point automatically to the class of abstract and parametrically simple models most often used. A problem for science is that there are so many heterogeneous models, and they are very hard to specify. 

The problem of evaluating the robustness of some policy boils down to figuring out some way of waving a wand and fitting the very large set of heterogeneous models to the data in some systematic way, and then turning each model around to predict the future. The prediction made from the ensemble of all those fitted heterogeneous models should be more robust than the simple one. MBITES is an honest stab at doing {\em that}. 

How far can we get if we just ignore heterogeneity? The cheeky answer is that we can do everything and be assured we will never find a better answer, so long as we never go looking for one. If we're honest about the peformance of predictive models for infectious diseases, we'd have to admit we haven't been very successful. When the problem comes to giving advice, we don't know if we're doing any better.  Obviously, we don't have a sound basis for ignoring heterogeneity, however inconvenient it might be. That is, in part, why MBITES exists. 

Since we can't wave our wands, we need a way of dealing with the daunting problem of heterogeneity, besides the easily accepted answer; ignore it. MBITES has taken a different approach. The flexibility built into MBITES makes it comparatively easy to configure a landscape with any degree of heterogeneity, but it creates a new problem of how to systematically explore heterogeneity.  Our approach is to start from the same point as others -- we begin by anchoring the simulation model to the case that most closely resembles the homogenous approximation, and then we steadily add heterogeneity. There is a logical progression: 
\begin{itemize}
\item If it's not necessary, turn it off. 
\item If it's on, set everything to a constant value. 
\item If a constant value won't do, try a line. 
\item If the value isn't the same everywhere, see have far you can get with the simplest way of incorporating noise.
\item If there's a pattern, base it on a case study.  
\end{itemize}
To put it another way, we honor William of Occam by preferring simple answers, in part, so we can say whether something we've observed can be explained with a simpler model. Our question, though, is not to explain the past but to know with some confidence that we are making robust policy recommendtions for the future. With these principles in mind, we've developed a set of functions and utilities to set up landscapes that roughly follow these rules. 

\clearpage 

\section{Core Behavioral Bouts}

MBITES is concerned with the basic problem of understanding the mosquito feeding and egg laying cycle: survival, the frequency of blood feeding, the proportion of meals taken on a human, and egg laying. To do this, MBITES considers each flight bout, and a basic question is how long does a mosquito rest between flight bouts. These are set by default for each one of the core behavioral bouts, which are:
\begin{itemize}
\item [{\bf F} ::] The blood meal search bout
\item [{\bf B} ::] The blood meal attempt bout
\item [{\bf R} ::] The post-prandial resting bout
\item [{\bf L} ::] The egg laying search bout
\item [{\bf O} ::] The egg laying attempt bout
\end{itemize}

\subsection{The Waiting Time between Bouts}

By default, the waiting time to the next bout varies by the mosquito's behavioral state. By default, all of these waiting times are exponentially distributed. The alternative is gamma distributed waiting times. 

\paragraph{Parsing}

For the behavioral states F,B,R,L,O, \& S, these are the timing options: 
%
\begin{itemize}
\item [{\bf 0}] \verb1boutX_tte_pdf=exponential1

  $$min + \mbox{rexp}(.,1/(mean-min))$$
  \begin{itemize}
  \item \verb1boutX_tte_mean1
  \item \verb1boutX_tte_min=01
%  \item \verb1boutX_tte_diurnal1
%    \begin{itemize}
%    \item [0] = FALSE 
%    \item [1] = TRUE
%      $$min + \mbox{rdiurnal}(.(mean-min), t, peak)$$
%      \begin{itemize} 
%      \item \verb1boutX_tte_peak=01 (midnight)
%      \item \verb1boutX_tte_amp=11 
%      \end{itemize}
%    \end{itemize}
  \end{itemize}
\item [1] \verb1boutX_tte_pdf=gamma1
  $$min + \mbox{rgamma}(.,mean-min, scale)$$
  \begin{itemize}
    \item \verb1boutX_tte_mean1
    \item \verb1boutX_tte_scale1
    \item \verb1boutX_tte_min=01
  \end{itemize}
  
\end{itemize}

\subsection{Survival}

How and where do mosquitoes die? The answer is that no one really knows, so we have made it possible for mosquitoes to die in every which way. MBITES has several options, but this creates a problem. Unlike other models, MBITES requires a person to make specific assumptions about how, where, and when mosquitoes die. In MBITES, mosquitoes can either die as a result of flight stress or from site-specific hazards associated with flight or landing. By default, site-specific hazards are all turned off, having no effect. The basic probability of dying is a bout-specific probability of surviving the stress of each flight, which is configured differently for each bout:
\begin{itemize}
\item \verb1boutX_surviveFlight1
\end{itemize}
The probability of dying as a result of flight stress can be modified so that it is related to a mosquito's chronological age, to flight energetics, or to any damage a mosquito has sustained over its life. Configuring the survival options requires some care, so we recommend running diagnostics after setting parameter values. This is done automatically in the MBITES-GUI. There are some basic differences in surviving flight stress compared with the site-specific hazards associated with surviving flight or landing.



\subsection*{[F] Blood Feeding Search Bout}

\begin{itemize}
\item \verb1boutF_surviveFlight_function=NULL1
  \begin{itemize}
  \verb1boutF_surviveFlight=.991
  \end{itemize}
\item \verb1boutF_surviveFlight_function="linearByDistance"1
  \begin{itemize}
  \item \verb1boutF_surviveFlight=1$.99- \psi\cdot \mbox{distance}$
  \end{itemize}
\end{itemize}


\subsection*{[B] Blood Feeding Attempt Bout}

default for blood meal size. 

\begin{itemize}
\item \verb1boutB_surviveFlight=0.991
\item \verb1chooseHost()1
\item \verb1bloodMealSize_function=NULL1
\end{itemize}

\paragraph{Human Blood Meal}


\begin{itemize}
\item \verb1boutB_surviveHumanApproach1
\item \verb1boutB_successHumanApproach1
\item \verb1boutB_surviveHumanProbing1
\item \verb1boutB_successHumanProbing1
\item \verb1boutB_surviveHumanBloodFeed1
\item \verb1boutB_successHumanBloodFeed1
\end{itemize}

\paragraph{Non-Human Blood Meal}

\begin{itemize}
\item \verb1boutB_surviveOther1
\item \verb1boutB_successOther1
\end{itemize}

\subsection*{[R] Post-Prandial Resting Bout}

\begin{itemize}
\item \verb1boutR_surviveFlight1
\item \verb1eggBatch_function=NULL1
  \begin{itemize}
  \item \verb1eggBatchSize1
  \item \verb1refeed_function=NULL1
    \begin{itemize}
    \item \verb1pr_Refeed1
    \end{itemize}
  \end{itemize}
  \item \verb1eggBatch_function!=NULL1 (Section 5, below)
\end{itemize}

\subsection*{[L] Egg Laying Search Bout}

\begin{itemize}
\item \verb1boutL_surviveFlight1
\item \verb1boutL_success1
\end{itemize}

\subsection*{[O] Egg Laying Attempt Bout and Oviposition}

\begin{itemize}
\item \verb1boutO_surviveFlight1
\item \verb1boutO_success1
\item \verb1boutO_skip=NULL1
\item \verb1boutO_skip="even"1
  \begin{itemize}
  \item \verb1skip_N1 
  \end{itemize}
\end{itemize}

\section{Pathogens}

\begin{itemize}
\item \verb1EIP_option = "constant"1
  \begin{itemize}
  \item \verb8EIP = 128
  \end{itemize}
\item \verb1EIP_option = "lookup"1
  \begin{itemize}
  \item \verb8EIP_file = <filename>8
  \end{itemize}
  
\end{itemize}

\section{Survival :: Configuring Options}

\subsection{Senescence}

\begin{itemize}
\item [0] \verb1senesce_function=NULL1
\item [1] \verb1senesce_function="gompertz"1
  \begin{itemize}
  \item \verb8senesce_gompertz_p18
  \item \verb8senesce_gompertz_p28
  \end{itemize}
\end{itemize}

\subsection{Damage}

The variable \verb1damage1 tracks all kinds of damage, including physical damage and contact with pesticides. Damage is cumulative,  and it is tracked on a scale of $[0,1)$. 

\begin{itemize}
\item [0] \verb1damage_function=NULL1
\item [1] \verb1surviveDamage_function="zzSigmoid"1
  \begin{itemize}
  \item \verb8damage_zzSigmoid_zero8
  \item \verb8damage_zzSigmoid_p508
  \item \verb8damage_zzSigmoid_slope8
  \end{itemize}
\end{itemize}

\paragraph{Tattering}

Tattering describes physical damage resulting from flight. 
\begin{itemize}
\item [0] \verb1tatter_function=NULL1
\item [1] \verb1tatter_function="zibeta"1
  \begin{itemize}
  \item \verb8tatter_zibeta_p08
  \item \verb8tatter_zibeta_mean8
  \item \verb8tatter_zibeta_ss8
  \end{itemize}
\end{itemize}

\subsection{Flight Energetics}

The variable \verb1energy1 tracks a mosquito's energy reserves on a scale of $(0,1]$. A mosquito is dead if it ever has $0$ energy units. Basic energy use is configured by describing what proportion of energy is used during a flight. For convenience, this is the inverse of the total number of flights a mosquito could take if it started with a full load and never refueled, $EU$.  For example, after topping up to full and taking $n$ flights, a mosquito's energy would have the value $1-n/EU$. 

\begin{itemize}
\item \verb1flightEnergetics=FALSE1
\item \verb1flightEnergetics=TRUE1
  \begin{itemize}
  \item \verb1flightEnergetics_EU1 :: Energy use per flight, a number between $0$ and $1$. The inverse of this number is approximately the number of flights a mosquito can take before it runs out of energy (see above). 
  \item \verb8flightEnergetics_bloodTopUp_EU=08 :: Setting it to $x>0$ means a mosquito is able to take $x$ flights as a result of a full blood meal. 
  \item \verb1flightEnergetics_survival=NULL1 :: The mosquito dies only if its energy state falls below.
  \item \verb1flightEnergetics_survival="zzSigmoid"1 :: Survival declines to zero as the energy state approaches zero. 
    \begin{itemize}
    \item \verb8damage_zzSigmoid_zero8
    \item \verb8damage_zzSigmoid_p508
    \item \verb8damage_zzSigmoid_slope8
    \end{itemize}
  \end{itemize}
\end{itemize}

\section{Sugar Feeding}

Sugar feeding builds on flight energetics. 

\begin{itemize}
\item \verb1sugarFeeding = FALSE1
\item \verb1sugarFeeding = TRUE1
  \begin{itemize}
  \item \verb8flightEnergetics_sugarTopUp_EU=08 :: Setting it to $x>0$ means a mosquito is able to take $x$ flights as a result of a full blood meal.
  \item \verb1sugarFeeding_boutS=FALSE1
  \item \verb1sugarFeeding_boutS=TRUE1 (see {\em Sugar Feeding Bout}, below)
  \item \verb1sugarFeeding_opportunistic=FALSE1
  \item \verb1sugarFeeding_opportunistic=TRUE1 (see {\em Opportunistic  Sugar Feeding}, below)
  \end{itemize}
\end{itemize}

\subsection*{[S] Sugar Feeding Bout}

\begin{itemize}
\item \verb1sugarFeedingBout=TRUE1
  \begin{itemize}
  \item \verb1boutS_survive1
  \item \verb1initSugarFeedingBout_function = "zzSigmoid"1
    \begin{itemize}
    \item \verb1initSugarFeedingBout_pzero1
    \item \verb1initSugarFeedingBout_p501
    \item \verb1initSugarFeedingBout_slope1
    \end{itemize}
  \end{itemize}
\end{itemize}

\subsection*{Opportunistic Sugar Feeding}

\begin{itemize}
\item \verb1sugarFeeding_opportunistic=TRUE1
  \begin{itemize}
  \item \verb1grabSugar_function="zzSigmoid"1
    \begin{itemize}
    \item \verb1grabSugar_pzero1
    \item \verb1grabSugar_p501
    \item \verb1grabSugar_slope1
    \end{itemize}
  \end{itemize}
\end{itemize}

\section{Blood Meal and Egg Batch}

\begin{itemize}
\item \verb1eggBatch_function=NULL1
  \begin{itemize}
  \item \verb1eggBatchSize1
  \item \verb1refeed_function=NULL1
    \begin{itemize}
    \item \verb1pr_Refeed1
    \end{itemize}
  \end{itemize}
\item \verb1eggBatch_function="proportional"1
  \begin{itemize}
  \item \verb1bloodMealSize_function="beta"1
    \begin{itemize}
    \item \verb1bloodMealSize_mean1
    \item \verb1bloodMealSize_ss1
    \item \verb1maxEggBatch1
  \end{itemize}
  \item \verb3refeed_function="1-zzSigmoid"3
    \begin{itemize}
    \item \verb1refeed_pzero1
    \item \verb1refeed_p501
    \item \verb1refeed_slope1
    \end{itemize}
  \end{itemize}
  \item \verb1eggBatch_function = "Aedes"1
    \begin{itemize}
    \item \verb1bloodMealSize_function="beta"1
    \begin{itemize}
    \item \verb1bloodMealSize_mean1
    \item \verb1bloodMealSize_ss1
    \item \verb1maxEggBatch1
    \end{itemize}
     \item \verb3refeed_function="fullBatch"3
      \begin{itemize}
      \item \verb1blood4FullBatch=x1 :: The number of full bloodmeals (or equivalent, usually $x >1$), required for a full batch of eggs. 
      \end{itemize}
  \end{itemize}
\end{itemize}




\section{Maturation} 

\begin{itemize}
\item \verb1bornMature=TRUE1 
  \begin{itemize}
  \item \verb1Mating=FALSE1   
  \end{itemize}
\item \verb1bornMature=FALSE1 
  \begin{itemize}
  \item \verb1maturationEU=01 : Set to some other value to require energy.
  \item \verb1Mating=TRUE1 (see Options below)
  \end{itemize}
\end{itemize}

\subsection{Mating and Males}
\begin{itemize}
\item \verb1Mating = FALSE1
\item \verb1Mating = TRUE1
  \begin{itemize}
  \item \verb1maleMosquito="kernel"1
      \begin{itemize}
      \item \verb1maleDeathRate1 
      \item \verb1matingKernel1
      \end{itemize}
  \item \verb1maleMosquito="ibm"1
     \begin{itemize}
     \item \verb1maleFlightEnergetics1 : mirrors \verb1flightEnergetics1 for females 
     \item \verb1maleSugarFeeding=FALSE1
     \item \verb1maleSugarFeeding=TRUE1 :: mirrors \verb1sugarFeeding1 for females
     \item \verb1swarming=TRUE1 
%     \item opportunistic
%        \begin{itemize}
%        \item males
%        \item females
%        \end{itemize}
     \end{itemize}
  \end{itemize}
\end{itemize}

\subsection*{[M] Mating Bout}

\begin{itemize}
\item \verb1swarming=FALSE1
\item \verb1swarming=TRUE1
  \begin{itemize}
     \item \verb1boutM_tte_function="matingTrigger"1 
        \begin{itemize}
         \item \verb1timeToSwarm1 
         \end{itemize}
     \item \verb1boutM=TRUE1 :: see below
       \begin{itemize}
       \item \verb1boutM_survive1
       \item \verb1boutM_success1
       \end{itemize}
     \item \verb1maleBoutM=TRUE1
        \begin{itemize}
          \item \verb1maleBoutM_survive1
          \item \verb1maleBoutM_success1
        \end{itemize}
  \end{itemize}
\item \verb1opportunisticMating=FALSE1
\end{itemize}

To do: \verb1opportunisticMating="Aedes"1, following Tom's description of males approaching females at blood feeding sites as they try to feed. 


\section{Types and Inheritance}

\begin{itemize}
\item \verb1resistantTypes=NULL1
\item \verb1gmClassXTypes=NULL1
\end{itemize}

\section{Estivation}

\begin{itemize}
\item \verb1estivation=FALSE1
\item \verb1estivation=TRUE1
  \begin{itemize}
  \item \verb1boutE_survive1
  \item \verb1boutE_nextState = "F"1
  \item \verb1Estivate_tte="seasonal"1
    \begin{itemize}
    \item initEstivation function 
    \item endEstivation function
    \end{itemize}
  \end{itemize}
\end{itemize}

\section{Vector Control}


\section{Diagnostics and Visualization}



\end{document}
