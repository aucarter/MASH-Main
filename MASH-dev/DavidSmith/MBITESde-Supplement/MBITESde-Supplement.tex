\documentclass{article}
\usepackage{mathtools}
\renewcommand{\baselinestretch}{1.15} 
\usepackage{Sweave}
\begin{document}
\Sconcordance{concordance:MBITESde-Supplement.tex:MBITESde-Supplement.Rnw:%
1 3 1 1 0 283 1}

\newcommand{\eg}{{\em e.g., }}
\newcommand{\ie}{{\em i.e., }}


\section{Supplementary Methods: MBDETES}

Let the behavioral state variable name from MBITES denote the expected number of mosquitoes of a given chronological age ($a$), that are in each behavioral state; \eg the proportion in the post-prandial resting state (R) of age $a$ at time $t$ is $R(a,t)$. Similarly, let the waiting times to events be modeled as a rate that is dependent on the behavioral state ($x$), age, and time ($t$): $\xi_x(a,t)$. The proportion of mosquitoes that transition from state $x$ to state $y$ at the end of a bout is denoted $P_{xy}$. Death rates can be age dependent (\ie due to senescence), which affects the proportions transitioning to other states, so we write $P_{xy}(a)$. To deal with the event-driven nature of these bouts and the possibility some bouts may be repeated many times before transitioning to another state, we index mosquitoes by the $i^{th}$ attempt to repeat the same event as a way of computing waiting times properly; for example, a mosquito repeating a blood feeding attempt bout transitions from $B_n(a)$ to $B_{n+1}(a)$. Finally, we let $\Lambda(t)$ represent the rate of emergence of adult female mosquitoes. The following system of coupled PDEs is homologous to MBITES: 

\begin{equation}\begin{array}{rl}
F_1(0,t)=& \Lambda(t) \\ &\\
%
\frac{\partial F_1(a,t)}{\partial t} + \frac{\partial F_1(a,t)}{\partial a} =& 
\xi_O(a,t) P_{OF}(a) \sum_i O_i(a,t) 
+ \xi_B(a,t) P_{BF}(a)  \sum_i B_i(a,t) \\&
+ \xi_R(a,t) P_{RF}(a) R(a,t) 
- \xi_F(a,t) F_1(a,t) \\ &\\
%
\frac{\partial F_i(a,t)}{\partial t} + \frac{\partial F_i(a,t)}{\partial a} =& \xi_F(a,t) P_{FF}(a) F_{i-1} (a,t) - \xi_F(a,t) F_i(a,t) \\ &\\
%
%
\frac{\partial B_1(a,t)}{\partial t} + \frac{\partial B_1(a,t)}{\partial a} =&  
\xi_O(a,t) P_{OB}(a) \sum_i O_i(a,t) 
+ \xi_F(a,t) P_{FB}(a) \sum_i F_i(a,t)\\& 
+ \xi_R(a,t) P_{RB}(a) R(a,t)
\\&
- \xi_B(a,t) B_1(a,t) \\ &\\
%
\frac{\partial B_i(a,t)}{\partial t} + \frac{\partial B_i(a,t)}{\partial a} =& \xi_B(a,t) P_{BB}(a) B_{i-1}(a,t) -
\xi_B(a,t) B_i(a,t) 
\\&\\
%
\frac{\partial R(a,t)}{\partial t} + \frac{\partial R(a,t)}{\partial a} =&  \xi_B(a,t) P_{BR}(a) \sum_i B_i(a,t) - \xi_R(a,t) R(a,t)\\ &\\
%
\frac{\partial L_1(a,t)}{\partial t} + \frac{\partial L_1(a,t)}{\partial a} =& \xi_R(a,t) P_{RL}(a) R(a,t) + 
\xi_O(a,t) P_{OL}(a) \sum_i O_i(a,t)\\& - \xi_L(a,t) L_1(a,t)
\\ &\\
%
\frac{\partial L_i(a,t)}{\partial t} + \frac{\partial L_i(a,t)}{\partial a} =& \xi_L(a,t) P_{LL}(a) L_{i-1} (a,t) - \xi_L(a,t) L_i(a,t)
\\ &\\
%
\frac{\partial O_1(a,t)}{\partial t} + \frac{\partial O_1(a,t)}{\partial a} =& \xi_L(a,t) P_{LO}(a) \sum_i L_i(a,t) 
+ \xi_R(a,t) P_{RO}(a) R(a,t) \\&
- \xi_O(a,t) O_1(a,t)
\\ &\\
%
\frac{\partial O_i(a,t)}{\partial t} + \frac{\partial O_i(a,t)}{\partial a} =& \xi_O(a,t) P_{OO}(a) O_{i-1} (a,t)   - \xi_O(a,t) O_i(a,t)
\\ 
%
\end{array}\end{equation}

It is a nuisance to deal with an infinite set of equations, but
if the state transitions are Markovian, then a change of
variables to lump the the $n^{th}$ states together: $x = \sum_i
x_i$, with a rescaled rate variable, $\gamma_x(a,t) =
\xi(a,t) (1-P_{xx})$   

\begin{bf}
Proof 1
\end{bf}

To justify this summation, consider the infinite set of equations
$$\frac{d x_1}{dt} = -\lambda x_1$$
$$\frac{d x_i}{dt} = p \lambda x_{i-1} - \lambda x_i$$
with initial conditions $x_1(0) = 1$, $x_i(0) = 0$. That is, initially
all of the mosquitoes are in their first attempt for an exponentially
distributed length of time. A proportion $p$ are successful or leave frustrated,
and $1-p$ attempt again. This system can be solved iteratively; $x_1$ has the solution
$$x_1(t) = e^{-\lambda t}$$ 
which, when plugged into the equation for $x_2$, gives
$$\frac{d x_2}{dt} = p \lambda e^{-\lambda t} - \lambda x_2$$ 
which can be solved using an integrating factor. This yields
$$x_2(t) = p \lambda t e^{-\lambda t}$$
This appears to be a weighted gamma distribution, which motivates the
assumption for an inductive-step solution of
$$x_i(t) = \frac{(p \lambda t)^{i-1}}{(i-1)!} e^{-\lambda t}$$
Assuming this, we look at the $x_{i+1}$ equation:
$$\frac{d x_{i+1}}{dt} = p\lambda x_i - \lambda x_{i+1}$$
Plugging in the assumed solution, we get
$$\frac{d x_{i+1}}{dt} = p\lambda \frac{(p \lambda t)^{i-1}}{(i-1)!} e^{-\lambda t} - \lambda x_{i+1}$$
$$= \frac{(p \lambda)^i}{i!}t^{i-1} e^{-\lambda t} - \lambda x_{i+1}$$
Which is again amenable to an integrating factor combined with i-1 integration by parts,
yielding
$$x_{i+1} = \frac{(p \lambda t)^{i}}{i!} e^{-\lambda t}$$
which completes the induction.
Because we are interested in the total amount of time spent and not the time
spent in any one compartment, we add the solutions together:
$$\sum_{i = 1}^{\infty} x_i(t) = \sum_{i = 1}^{\infty}\frac{(p \lambda t)^{i-1}}{(i-1)!} e^{-\lambda t}$$
$$ = e^{-\lambda t} \sum_{i=0}^{\infty}\frac{(p \lambda t)^i}{i!}$$
$$ = e^{-\lambda t} e^{p\lambda t}$$
$$ = e^{-(1-p) \lambda t}$$

Normalizing this gives us the total expected waiting time in this state for the
mosquito is exponential with intensity $(1-p) \lambda$.

As a note, this convergence is uniform as it is the Taylor series representation
of the exponential function - this justifies summing the infinite equations and
the differential operator which gives us

$$\frac{d}{dt} \sum_{i=1}^{\infty} x_i = \sum_{i=1}^{\infty}\frac{d x_i}{dt}$$
$$ = \sum_{i=1}^{\infty}-\lambda (1-p) x_i$$

setting $X = \sum_{i=1}^{\infty}x_i$, we get a very simple single differential equation

$$\frac{dX}{dt} = - (1-p) \lambda X$$

with the initial condition $X(0) = 1$. This has the same solution we found through induction.

\begin{bf}
Proof 2
\end{bf}

Say we want to find the total waiting time T a mosquito spends in a given state.
Using the law of total probability, we can condition this on the number of attempts N
a mosquito will make:
$$P(T=t) = \sum_{n=1}^{\infty} P(T=t| N=n) P(N=n)$$
The number of attempts is geometrically distributed, as it will succeed or give up
with probability (1-p) and therefore try again with probability p - it will try until
it succeeds or gives up. Therefore
$$P(N=n) = (1-p)p^{n-1}$$
The waiting time between each attempt is iid with an exponentially distributed waiting
time with intensity $\lambda$, so given it takes n trials the distribution follows
a gamma distribution:
$$T|N = \sum_{i=1}^{n}Exp_i(\lambda)$$
$$ \sim \Gamma (n,\lambda)$$

Therefore we have

$$P(T=t) = \sum_{i=1}^{\infty}P(T=t|N=n)P(N=n)$$
$$ = \sum_{i=1}^{\infty}(1-p)p^{n-1}\frac{\lambda^n}{\Gamma (n)}t^{n-1} e^{-\lambda t}$$
$$ = (1-p) \lambda e^{-\lambda t} \sum_{i=0}^{\infty} \frac{(p\lambda t)^n}{n!}$$
$$ = \lambda (1-p) e^{-\lambda t} e^{p\lambda t}$$
$$ = (1-p) \lambda e^{-(1-p) \lambda t} $$

which is exactly the normalized solution to the previous system of ordinary differential
equations - the waiting time for the mosquito to leave the state is exponentially 
distributed with intensity $(1-p) \lambda$. Importantly, this proof does not depend on
$(1-p)$ or $\lambda$ being constant - they can be age- or time-dependent.

This means we can rewrite the infinite system of equations as a
set of 5 differential equations: 
%
\begin{equation}\begin{array}{rl}
F(0,t) =& \Lambda(t) \\ 
%
&\\
\frac{\partial F(a,t)}{\partial t} + \frac{\partial F(a,t)}{\partial a} =& 
\gamma_O(a,t) P_{OF}(a) O(a,t) 
+ \gamma_B(a,t) P_{BF}(a) B(a,t) \\&
 + \gamma_R(a,t) P_{RF}(a) R(a,t)
- \gamma_F(a,t) (1-P_{FF}(a)) F(a,t) \\
%
%
&\\
\frac{\partial B(a,t)}{\partial t} + \frac{\partial B(a,t)}{\partial a} =&  \gamma_O(a,t) P_{OB}(a) O(a,t) + \gamma_F(a,t) P_{FB}(a) F(a,t) \\ & 
+ \gamma_R(a,t) P_{RB}(a) R(a,t) 
- \gamma_B(a,t) (1-P_{BB}(a)) B(a,t)\\
%
&\\
\frac{\partial R(a,t)}{\partial t} + \frac{\partial R(a,t)}{\partial a} =&  \gamma_B(a,t) P_{BR}(a) B(a,t) - \gamma_R(a,t) R(a,t)\\ 
%
&\\
\frac{\partial L(a,t)}{\partial t} + \frac{\partial L(a,t)}{\partial a} =& \gamma_R(a,t) P_{RL}(a) R(a,t) + 
\gamma_O(a,t) P_{OL}(a) O(a,t) \\&
- \gamma_L(a,t) (1-P_{LL}(a)) L(a,t)
\\ 
%
&\\
\frac{\partial O(a,t)}{\partial t} + \frac{\partial O(a,t)}{\partial a} =& \gamma_L(a,t) P_{LO}(a) L(a,t) 
+ \gamma_R(a,t) P_{RO}(a) R(a,t)  
\\&
- \gamma_O(a,t)(1-P_{OO}(a)) O(a,t)

%
\end{array}\end{equation}


\subsection{The MBITES-de for Cohorts}

Finally, we want a version of these equations to model changes in a cohort of individuals with respect to age (assuming all the mosquitoes emerge from aquatic habitats at the same time of day): 
%
\begin{equation}\begin{array}{rcl}
F(0) &=& 1 \\ 
%
{\dot F}&=& 
\gamma_O(a) P_{OF}(a) O(a) 
+ \gamma_B(a) P_{BF}(a) B(a) 
 + \gamma_R(a,t) P_{RF}(a) R(a)
\\ &&
- \gamma_F(a) (1 - P_{FF}(a)) F(a) 
 \\ 
%
%
{\dot B} &=&  \gamma_O(a) P_{OB}(a) O(a) + \gamma_F(a) P_{FB}(a) F(a) 
+ \gamma_R(a) P_{RB}(a) R(a) \\ && 
- \gamma_B(a) (1-P_{BB}(a)) B(a)
\\
%
{\dot R} &=&  \gamma_B(a) P_{BR}(a) B(a) - \gamma_R(a) R(a)\\ 
%
{\dot L}&=& \gamma_R(a) P_{RL}(a) R(a) + 
\gamma_O(a) P_{OL}(a) O(a) \\&&
- \gamma_L(a) (1 - P_{LL}(a)) L_e(a)
\\ 
%
{\dot O} &=& \gamma_L(a) P_{LO}(a) L(a) 
+ \gamma_R(a) P_{RO}(a) R(a)  
\\&& 
- \gamma_O(a) (1 - P_{OO}(a)) O(a)
\\ 
%
\end{array}\end{equation}

\subsection{Infection Dynamics in the MBITES-de Equations}

To simulate infection dynamics in MBITES-de, we subdivide each variable $X$ into new variables $X_x$, $x \in \left\{ U, Y, Z \right\}$, to represent the fraction of mosquitoes in behavioral state $X$ that are uninfected, $U$, infected, $Y$, or infected and infectious $Z$.  These lead to the following systems of coupled differential equations that remain unchanged, but for the equation describing resting mosquitoes. We let  $Q\kappa(t)$ the proportion of mosquitoes becoming infected after blood feeding at time $t$. 
%
\begin{equation}\begin{array}{rl}
%
\frac{\partial R_U(a,t)}{\partial t} + \frac{\partial
R_U(a,t)}{\partial a} =& \left(1-Q \kappa(t) \right) \gamma_B(a,t) P_{BR}(a) B_U(a,t)
- \gamma_R(a,t) R_U(a,t)\\

\frac{\partial R_Y(a,t)}{\partial t} + \frac{\partial
R_Y(a,t)}{\partial a} =&  Q \kappa(t) \gamma_B(a,t) P_{BR}(a) B_U(a,t)\\&
+ \gamma_B(a,t) P_{BR}(a) B_Y(a,t) 
- \gamma_R(a,t) R_Y(a,t)\\
\end{array}\end{equation}
%
We let $\tau(t)$ denote the (possibly time-dependent) extrinsic incubation period. Because $\tau(t)$ is time dependent, we let $\hat t$ denote that point in the past when the mosquito became infected in order to become infectious at time $t$: \ie $t = \hat t + \tau(\hat t)$. Let $\rho(t)$ the proportion of mosquitoes surviving through the extrinsic incubation period (\ie from $\hat t$ to $t = \hat t +\tau (\hat t)$). An equation describing the proportion of infectious mosquitoes is:
%
\begin{equation}\begin{array}{rl}
\frac{\partial R_Z(a,t)}{\partial t} + \frac{\partial
R_Z(a,t)}{\partial a} =&  \rho(t) Q \kappa \left( \hat t \right) \gamma_B(a,t) P_{BR}(a) B_U(a,t) \\&
+ \gamma_B(a,t) P_{BR}(a) B_Z(a,t)
- \gamma_R(a,t) R_Z(a,t)\\
%
\end{array}\end{equation}


The remaining equations remain as follows: 
%
\begin{equation}\begin{array}{rl}
F_{x}(0,t) =& \Lambda(t) \\
%
\frac{\partial F_{x}(a,t)}{\partial t} + \frac{\partial
F_{x}(a,t)}{\partial a} =&
\gamma_O(a,t) P_{OF}(a) O_x(a,t)
+ \gamma_B(a,t) P_{BF}(a) B_x(a,t)\\ &
 + \gamma_R(a,t) P_{RF}(a) R_x(a,t)
- \gamma_F(a,t) (1-P_{FF}(a)) F_{x}(a,t)\\
%
%
\frac{\partial B_{o,x}(a,t)}{\partial t} + \frac{\partial
B_{x}(a,t)}{\partial a} =&  \gamma_O(a,t) P_{OB}(a) O_x(a,t) +
\gamma_F(a,t) P_{FB}(a) F(a,t)\\ &
+ \gamma_R(a,t) P_{RB}(a) R_x(a,t) 
- \gamma_B(a,t) (1 - P_{BB}(a)) B_{x}(a,t)\\
%

\frac{\partial L_{x} (a,t)}{\partial t} + \frac{\partial
L_{x} (a,t)}{\partial a} =& \gamma_R(a,t) P_{RL}(a) R_x(a,t) +
\gamma_O(a,t) P_{OL}(a) O_x(a,t) \\&
- \gamma_L(a,t) (1 - P_{LL}(a)) L_{x}(a,t)
\\
%
\frac{\partial O_{o,x}(a,t)}{\partial t} + \frac{\partial
O_{x}(a,t)}{\partial a} =& \gamma_L(a,t) P_{LO}(a) L(a,t)
+ \gamma_R(a,t) P_{RO}(a) R_x(a,t)
\\&
- \gamma_O(a,t) (1 - P_{OO}(a)) O_{x}(a,t)
\\
%
\end{array}\end{equation}
%







\end{document}
